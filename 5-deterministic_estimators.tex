\chapter{Probability in Practice: Deterministic Estimators}

If we can't compute expectations with respect to our target distribution
analytically, one immediate strategy is to just replace it with a simpler one.  
In other words, we can approximate a complex target distribution, $\PP$, 
with a simpler distribution, $\widetilde{\PP}$, whose expectations, or at 
least some expectations of practical interest, are known analytically,
%
\begin{equation*}
\EE_{\PP} \! \left[ f \right] 
\approx 
\EE_{\widetilde{\PP}} \! \left[ f \right].
\end{equation*}

\emph{Deterministic estimators} use various criteria to identify an 
optimal approximating distribution so that expectations can be 
approximated deterministically.

\section{Modal Estimators}

Ideally all expectations with respect to our approximating distribution 
would be analytic so that we could use it to approximate any 
expectation with respect to our target distribution.  The only
probability distribution that fits this criteria is the \emph{Dirac distribution},
$\mathbb{D}_{\tilde{\theta}}$, that assigns all probability to a single point 
in the sample space, $\tilde{\theta}$,
%
\begin{equation*}
\mathbb{D}_{\tilde{\theta}} \! \left[ E \right] = 
\left\{
\begin{array}{rr}
0, & \tilde{\theta} \notin E  \\
1, & \tilde{\theta} \in E
\end{array}
\right. .
\end{equation*}
%
Because all probability concentrates at $\tilde{\theta}$, expectations 
are trivial,
%
\begin{equation*}
\EE_{\mathbb{D}} \! \left[ f \right] 
=
 f \! \left( \tilde{\theta} \right).
\end{equation*} 

Where, however, should we assign all probability to best approximate
the target distribution and its typical set?  One of the simplest, and 
consequently most popular, deterministic estimation strategies is 
\emph{modal estimation}, where we approximate the target distribution 
with a Dirac distribution at the mode of the probability density function,
%
\begin{equation*}
\hat{\theta} = \mathrm{argmax} \, p \! \left( \theta \right).
\end{equation*}
%
This approach, however, immediately contradicts the intuition provided by 
concentration of measure: it utilizes a single point in the sample space
that lies outside of the typical set and depends entirely on the choice of
probability density function representation!  Why, then, is modal estimation
so ubiquitous?

Modal estimators are seductive because the optimization on which they
rely is relatively computationally inexpensive.  Moreover, in some very 
simple cases modal estimators constructed from the probability density function
in \emph{some} sample spaces can be reasonably accurate for 
\emph{some} functions.  For example, if the target probability distribution 
is sufficiently simple that the typical set is convex \emph{and} if a 
sample space can be found such that the mode of the probability density 
function lies in the center of that typical set, then the corresponding modal 
estimator might yield reasonably accurate estimates for the mean, 
$\EE_{\PP_{\Theta}} \! \left[ \theta \right]$ (Figure \ref{fig:good_map_bad_map}).

\begin{figure*}
\centering
\subfigure[]{
\begin{tikzpicture}[scale=0.3, thick]

  \begin{scope}
    \clip (-12, -6) rectangle (12, 10);
    \foreach \i in {0, 0.05,..., 1} {
      \draw[line width={30 * \i}, opacity={exp(-8 * \i)}, dark] 
        (-10, 0) .. controls (-10, 15) and (-5, 5) .. (0, 5)
        .. controls (5, 5) and (10, 8) .. (10, 0)
        .. controls (10, -8) and (5, -3) .. (0, -3)
        .. controls (-5, -3) and (-10, -5) .. (-10, 0);
    }  
  \end{scope}
  
  \fill[color=dark] (0, 1) circle (7pt);  
  \fill[color=light] (0, 1) circle (5pt)
  node[left, color=black]  { $\EE_{\PP} \! \left[ \theta \right]$ };
 
  \fill[color=dark] (1, 0.5) circle (7pt);
  \fill[color=green] (1, 0.5) circle (5pt)
  node[right, color=black]  { $\hat{\theta}$ };
  
\end{tikzpicture}
}
\subfigure[]{
\begin{tikzpicture}[scale=0.3, thick]

  \begin{scope}
    \clip (-12, -6) rectangle (12, 10);
    \foreach \i in {0, 0.05,..., 1} {
      \draw[line width={30 * \i}, opacity={exp(-8 * \i)}, dark] 
        (-10, 0) .. controls (-10, 15) and (-5, 5) .. (0, 5)
        .. controls (5, 5) and (10, 8) .. (10, 0)
        .. controls (10, -8) and (5, -3) .. (0, -3)
        .. controls (-5, -3) and (-10, -5) .. (-10, 0);
    }  
  \end{scope}
  
  \fill[color=dark] (0, 1) circle (7pt);  
  \fill[color=light] (0, 1) circle (5pt)
  node[left, color=black]  { $\EE_{\PP} \! \left[ \theta \right]$ };
 
  \fill[color=dark] (5, 0.5) circle (7pt);
  \fill[color=green] (5, 0.5) circle (5pt)
  node[right, color=black]  { $\hat{\theta}$ };
  
\end{tikzpicture}
}
\caption{(a) In simple cases a prescient choice of sample space can 
yield a modal estimator that well approximates some expectations, 
such as the mean of the real parameters $\theta$.  (b) Poor choices 
of the representation, however, yield very inaccurate estimates, even 
in these simple problems.
}
\label{fig:good_map_bad_map}
\end{figure*}

Because they rely on a point estimate, however, modal estimators are 
terrible at approximating expectations that depend on the breadth of 
the typical set, such as the variance.  Furthermore, identifying the optimal 
sample space for a particular target function, even if one exists, is 
extremely challenging in practice.  Worse, we have no generic means
of even quantifying the error in these estimators for a generic target
distribution.  Ultimately this strong sensitivity to the choice of a particular
expression and the inability to validate the accuracy of the estimators 
makes modal estimation extremely fragile in practice.

\section{Laplace Estimators}

The fragility of modal estimators can be partially resolved by generalizing
them to \emph{Laplace estimators} which approximate the probability
density function with a Gaussian density function,
%
\begin{equation*}
p \! \left( \theta \right) \approx 
\mathcal{N} \! \left( \theta \mid \mu, \Sigma \right),
\end{equation*}
%
where the mean is given by the modal estimate,
%
\begin{equation*}
\mu = \hat{\theta},
\end{equation*}
%
and the covariance is given by the Hessian of the probability density 
function,
%
\begin{equation*}
\left( \Sigma^{-1} \right)_{ij} = 
\frac{ \partial^{2} }{ \partial \theta_{i} \partial \theta_{j} }
p \! \left( \theta \right).
\end{equation*}
%
Expectations are then estimated with Gaussian integrals
%
\begin{equation*}
\mathbb{E}_{\PP} \! \left[ f \right]
\approx 
\int_{\Theta} f \! \left( \theta \right) 
\mathcal{N} \! \left( \theta \mid \mu, \Sigma \right) \dd \theta,
\end{equation*}
%
which often admit analytic solutions.

The accuracy of Laplace approximations depends on how well the
mode and and the Hessian of the probability density function
quantify the geometry of the typical set (Figure \ref{fig:laplace}).
Unfortunately the conditions that are necessary for these estimates
to be reasonably accurate hold only for very simple probability
distributions, and then only if an appropriate sample space can be 
found.  

\begin{figure*}
\centering
\begin{tikzpicture}[scale=0.35, thick]

  \begin{scope}
    \clip (-12, -6) rectangle (12, 10);
    \foreach \i in {0, 0.05,..., 1} {
      \draw[line width={30 * \i}, opacity={exp(-8 * \i)}, dark] 
        (-10, 0) .. controls (-10, 15) and (-5, 5) .. (0, 5)
        .. controls (5, 5) and (10, 8) .. (10, 0)
        .. controls (10, -8) and (5, -3) .. (0, -3)
        .. controls (-5, -3) and (-10, -5) .. (-10, 0);
    }  
    
    \foreach \i in {0, 0.05,..., 1} {
      \draw[line width={30 * \i}, opacity={exp(-8 * \i)}, green] 
        (1, 0.5) ellipse (10 and 5);
    }
  \end{scope}
 
  \fill[color=dark] (1, 0.5) circle (7pt);
  \fill[color=green] (1, 0.5) circle (5pt)
  node[right, color=black]  { $\hat{\theta}$ };
  
\end{tikzpicture}
\caption{For simple probability distributions with well-chosen
samples spaces, the local geometry around the mode of a 
probability density function can quantify the geometry of the 
entire typical set, yielding accurate Laplace estimators. For
more complex probability distributions, however, this local
information poorly quantifies the global geometry of the 
typical set and Laplace estimators suffer from large biases.
}
\label{fig:laplace}
\end{figure*}

As with the simpler modal estimators, the dependence on the particular
sample space manifests as fragility of the corresponding estimators
and we have no generic means of quantifying the error in practice.  
If we want robust estimation of probabilities and expectations then we 
need strategies that do not depend on these irrelevant properties.

\section{Variational Estimators}

In order to construct an approximation that is not sensitive to the choice 
of a particular sample space we need to frame the problem as an 
optimization over a space of approximating distributions directly.
Optimizations over spaces of probability distributions fall into a class 
of algorithms known as \emph{variational methods}.

Variational methods are characterized by two choices: the variational
family and a divergence function. The \emph{variational family}, 
$\mathcal{Q}$, is a set of probability distributions over the target sample 
space, $\Theta$, such that at least some expectations can be computed
analytically.  In order to identify the best approximation to the target 
distribution we then define a \emph{divergence function},
%
\begin{align*}
D &: 
\mathcal{Q} \times \mathcal{Q}
\rightarrow \mathbb{R}^{+}
\\
& \quad \PP_{1}, \PP_{2} 
\mapsto D \! \left( \PP_{1} \mid\mid \PP_{2} \right),
\end{align*}
%
which is zero if the two arguments are the same and increases as they 
deviate from each other more strongly.

The best approximating distribution is then defined by the variational 
objective (Figure \ref{fig:variational_cartoon}).
%
\begin{equation*}
\mathbb{Q}^{*}
= 
\underset{\mathbb{Q} \in \mathcal{Q}}{\mathrm{argmin}} \,
D \! \left( \PP \mid\mid \mathbb{Q} \right).
\end{equation*}
%
Although straightforward to define, this variational optimization can be
quite challenging in practice.  Depending on how the target distribution
interacts with the choice of variational distribution and divergence
function, the variational objective might feature multiple critical points
and we may not be able to find the global optimum in practice
(Figure \ref{fig:variational_challenges}).

\begin{figure*}
\centering
%
\subfigure[]{
\begin{tikzpicture}[scale=0.225, thick]
  
  \draw[color=white] (-15, 0) -- (15, 0);
  
  \fill[mid] (0, 0) ellipse (13 and 7);
  \node at (11, -6) {$\mathcal{P}_{\Theta}$};
  
  \fill[light] (-4, 2) ellipse (6 and 3);
  \node[color=white] at (-2, -2.5) {$\mathcal{Q}$};
  
  \fill[color=white] (4, 2) circle (8pt)
  node[right, color=white] {$\PP$};
  
\end{tikzpicture}
}
\subfigure[]{
\begin{tikzpicture}[scale=0.225, thick]

  \draw[color=white] (-15, 0) -- (15, 0);
  
  \fill[mid] (0, 0) ellipse (13 and 7);
  \node at (11, -6) {$\mathcal{P}_{\Theta}$};
  
  \draw[color=light, dashed] (-4, 2) ellipse (6 and 3);
  \node[color=white] at (-2, -2.5) {$\mathcal{Q}$};
  
  \begin{scope}
    \clip (-4, 2) ellipse (6 and 3);
    \foreach \i in {0, 0.05,..., 1} {
      \fill[opacity={exp(-5 * \i*\i)}, light] (0, 2) circle ({4 * \i});      
    }
  \end{scope}
  
  \fill[color=white] (2, 2) circle (8pt)
  node[above, color=white] {$\mathbb{Q}^{*}$};
  
  \fill[color=white] (4, 2) circle (8pt)
  node[right, color=white] {$\PP$};
  
\end{tikzpicture}
}
\caption{(a) A variational family, $\mathcal{Q}$, is a set of probability distributions
taken from the set of all probability distributions over the same sample space, as 
the target distribution, $\mathcal{P}_{\Theta}$.  (b) Adding a divergence function 
distinguishes which elements of $\mathcal{Q}$ are good approximations to the 
target distribution, allowing us to identify the best approximation, $\mathbb{Q}^{*}$.
}
\label{fig:variational_cartoon}
\end{figure*}

\begin{figure*}
\centering
\begin{tikzpicture}[scale=0.225, thick]

  \draw[color=white] (-15, 0) -- (15, 0);
  
  \fill[mid] (0, 0) ellipse (13 and 7);
  \node at (11, -6) {$\mathcal{P}_{\Theta}$};
  
  \draw[color=light, dashed] (-4, 2) ellipse (6 and 3);
  \node[color=white] at (-2, -2.5) {$\mathcal{Q}$};
  
  \begin{scope}
    \clip (-4, 2) ellipse (6 and 3);
    
    \foreach \i in {0, 0.05,..., 1} {
      \fill[opacity={exp(-5 * \i*\i)}, light] (0, 2) circle ({4 * \i});      
    }
    
    \foreach \i in {0, 0.05,..., 1} {
      \fill[opacity={exp(-5 * \i*\i)}, light] (-4, 3) ellipse ({4 * \i} and {3 * \i});      
    }
    
    \foreach \i in {0, 0.05,..., 1} {
      \fill[opacity={exp(-5 * \i*\i)}, light] (-8, 1) ellipse ({4 * \i} and {5 * \i});      
    }
    
    \foreach \i in {0, 0.05,..., 1} {
      \fill[opacity={exp(-5 * \i*\i)}, light] (-3, -1) ellipse ({3 * \i} and {2 * \i});      
    }
  \end{scope}
  
  \fill[color=white] (2, 2) circle (8pt)
  node[above, color=white] {$\mathbb{Q}^{*}$};
  
  \fill[color=white] (4, 2) circle (8pt)
  node[right, color=white] {$\PP$};
  
\end{tikzpicture}
\caption{Typically different elements of the variational family are able to 
capture different characteristics of the target distribution and the variational 
objective manifests multiple optima.  Even if a global optimum, $\mathbb{Q}^{*}$, 
exists it will be difficult to find and we may be left with only a suboptimal local 
optimum.
}
\label{fig:variational_challenges}
\end{figure*}

Even if we could find the best approximating distribution, however,
there are no guarantees that it will yield accurate estimates for
all relevant expectations of our target distribution.  For example, 
some divergence functions are biased towards variational solutions
that underestimate the breadth of the typical set while others
tend to significantly overestimate it (Figure \ref{fig:variational_approximations}).

Variational methods are relatively new to statistics and at the moment
there are no generic methods for quantifying the error in variational
estimators.

\begin{figure*}
\centering
\subfigure[]{
\begin{tikzpicture}[scale=0.35, thick]
  \begin{scope}
    \clip (-12, -6) rectangle (12, 10);
    \foreach \i in {0, 0.05,..., 1} {
      \draw[line width={30 * \i}, opacity={exp(-8 * \i)}, dark] 
        (-10, 0) .. controls (-10, 15) and (-5, 5) .. (0, 5)
        .. controls (5, 5) and (10, 8) .. (10, 0)
        .. controls (10, -8) and (5, -3) .. (0, -3)
        .. controls (-5, -3) and (-10, -5) .. (-10, 0);
    }  
    
    \foreach \i in {0, 0.05,..., 1} {
      \draw[line width={30 * \i}, opacity={exp(-8 * \i)}, green] 
        (0, 1) ellipse (9.75 and 3.75);
    }
  \end{scope}
\end{tikzpicture}
}
\subfigure[]{
\begin{tikzpicture}[scale=0.35, thick]
  \begin{scope}
    \clip (-14, -8) rectangle (14, 14);
    \foreach \i in {0, 0.05,..., 1} {
      \draw[line width={30 * \i}, opacity={exp(-8 * \i)}, dark] 
        (-10, 0) .. controls (-10, 15) and (-5, 5) .. (0, 5)
        .. controls (5, 5) and (10, 8) .. (10, 0)
        .. controls (10, -8) and (5, -3) .. (0, -3)
        .. controls (-5, -3) and (-10, -5) .. (-10, 0);
    }  
    
    \foreach \i in {0, 0.05,..., 1} {
      \draw[line width={30 * \i}, opacity={exp(-8 * \i)}, green] 
        (0, 1.75) ellipse (12 and 9);
    }
  \end{scope}
\end{tikzpicture}
}
\caption{Intuition about the effects of a particular variational divergence 
function can be developed by considering how the typical set of an
approximating distribution (green) interacts with the typical of the target 
distribution (red).  (a) Some divergence functions favor approximating
distributions that expand into the interior of the target typical set,
resulting in an underestimate of the breadth of the typical set.  (b)
Others, however, favor approximating distractions that collapse around
the exterior of the target typical set, resulting in overestimated expectations.
}
\label{fig:variational_approximations}
\end{figure*}