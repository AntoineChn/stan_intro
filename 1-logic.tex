\chapter{Logic}

Before we can formalize \emph{learning} about a system we first need formal 
means of \emph{describing} that system.  \emph{Logical statements} provide a
mathematically-precise language for quantifying information about a system with 
certainty.  In this section we will review abstract logical statements, their manipulations, 
and how we can express them in practice.  We conclude with a brief discussion 
of implicative statements.

\subsection{Logical Statements}

Let $\mathcal{S}$ be an abstract system that we would like to describe.
Any description of $\mathcal{S}$ can be defined as the assertion of a logical 
statement about that system.  For example, consider describing one of the 
eight planets in our Solar System: Mercury ($\mercury$), Venus ($\venus$), 
Earth ($\earth$), Mars ($\mars$), Jupiter ($\jupiter$), Saturn ($\saturn$), 
Uranus ($\uranus$), and Neptune ($\neptune$).  Information about the planet 
is encapsulated by asserting the truth of logical statements such as ``the planet 
has an atmosphere'' or ``the planet has rings''.  We denote the set of well-posed
logical statements about $\mathcal{S}$ as $\lambda \! \left( \mathcal{S} \right)$.

Logical statements can also be manipulated into other logical statements.
A \emph{conjunction} asserts that two statements are both true, ``the planet 
has an atmosphere \emph{and} rings'' while a \emph{disjunction} states that 
either statement is true, ``the planet has an atmosphere \emph{or} rings''.  
We can also negate a logical statement by asserting its contrary, ``the planet 
does \emph{not} have rings''.

\subsection{Events}

Logical statements provide a generic framework for describing arbitrary and
abstract systems, but that abstraction also makes these statements ungainly 
to specify and manipulate in practice.  For our abstract descriptions to be useful 
we need an explicit language in which we can communicate them, which we
will refer to as an \emph{expression} (Table \ref{tab:expression_examples}).

\begin{table}
  \centering
  \renewcommand{\arraystretch}{1.5}
  \begin{tabular}{ccc}
    \rowcolor[gray]{0.9} 
    \textbf{Abstract System} & \textbf{Expression} & \textbf{Measurable Maps} \\
    %
    Historical Decrees & Ancient Languages & Translations \\
    & (Egyptian, Demotic, Green, $\ldots$) & (Rosetta Stone) \\
    % 
    Computer Program & Programming Languages & Compilers \\
    & (C++, S, Python, $\ldots$) & \\
    %
    Unordered Categories & Integers & Permutations \\
    %
    Direction & Vector Space & Changes of Basis \\
    %
    Numerical Quantification & Units & Changes of Units \\
    %
    Position & Coordinates & Reparameterizations \\
  \end{tabular}
\caption{Expressions provide a concrete representation of the abstract
descriptions of a given system.  Degenerate ways to express the 
descriptions of a system, along with the translations between them,
are ubiquitous in common experience.}
\label{tab:expression_examples}
\end{table}

An expression is a map from our abstract system into a concrete space,
such as the integers or the real numbers,
%
\begin{equation*}
\mathfrak{e} : \mathcal{S} \rightarrow \Theta,
\end{equation*}
%
where the concrete space, $\Theta$, is known as the \emph{sample space}.
An expression also defines a corresponding map from logical statements
into well-behaved subsets of the sample space, of \emph{events},
$E \subset \Theta$,
%
\begin{equation*}
\mathfrak{e}^{*} : 
\lambda \! \left( \mathcal{S} \right) \rightarrow \EV{\Theta},
\end{equation*}
%
where the \emph{event space}, $\EV{\Theta}$, collects all well-behaved
subsets of the sample space and hence all valid expressions of well-posed 
logical statements. Event spaces always include the null event, $E = \emptyset$, 
which represents certain falseness, and the trivial event, $E = \Theta$, which
represents certain truth.  The manipulation of the logical statements are 
implemented with in an expression with set operations.  Conjunction of 
statements becomes the intersection of subsets, disjunction the union of 
subsets, and negation the complement of a subset (Figure \ref{fig:venn_diagrams}).

\begin{figure*}
\centering
\subfigure[]{
\begin{tikzpicture}[scale=0.3, thick]
  \draw[color=black] (0, 0) to (20, 0) to (20, 15) to (0, 15) to (0, 0);
  \draw[color=gray80, dashed] (12, 9) circle (150pt); 
  \fill[color=dark] (8, 6) circle (150pt) node[color=white] {$E_{1}$}; 
\end{tikzpicture}
}
%
\subfigure[]{
\begin{tikzpicture}[scale=0.3, thick]
  \draw[color=black] (0, 0) to (20, 0) to (20, 15) to (0, 15) to (0, 0);
  \draw[color=gray80, dashed] (8, 6) circle (150pt); 
  \fill[color=dark] (12, 9) circle (150pt) node[color=white] {$E_{2}$}; 
\end{tikzpicture}
}
%
\subfigure[]{
\begin{tikzpicture}[scale=0.3, thick]
  \draw[color=black] (0, 0) to (20, 0) to (20, 15) to (0, 15) to (0, 0);
  \draw[color=gray80, dashed] (8, 6) circle (150pt); 
  \draw[color=gray80, dashed] (12, 9) circle (150pt); 
  \begin{scope}
    \clip (8, 6) circle (150pt);
    \fill[color=dark] (12, 9) circle (150pt); 
  \end{scope}
  \node[color=white] at (10, 7.5) {$E_{1} \cap E_{2}$};
\end{tikzpicture}
}
%
\subfigure[]{
\begin{tikzpicture}[scale=0.3, thick]
  \draw[color=black] (0, 0) to (20, 0) to (20, 15) to (0, 15) to (0, 0);
  \fill[color=dark] (8, 6) circle (150pt); 
  \fill[color=dark] (12, 9) circle (150pt); 
  \node[color=white] at (10, 7.5) {$E_{1} \cup E_{2}$};
\end{tikzpicture}
}
%
\subfigure[]{
\begin{tikzpicture}[scale=0.3, thick]
  \fill[color=dark] (0, 0) to (20, 0) to (20, 15) to (0, 15) to (0, 0);
  \draw[color=black] (0, 0) to (20, 0) to (20, 15) to (0, 15) to (0, 0);
  \fill[color=white] (8, 6) circle (150pt); 
  \node[color=white] at (16, 11) {$E_{1}^{c}$};
\end{tikzpicture}
}
%
\caption{In an expression logical statements are represented by
subsets and the manipulation of these statements is implemented
with set operations.  (a, b) If two statements are represented by
the sets $E_{1}$ and $E_{2}$, then (c) their conjunction is the
set intersection,and (d) their disjunction is the set union.  (e) Simlarly,
the negation of $E_{1}$ is the set complement.}
\label{fig:venn_diagrams}
\end{figure*}

For example, our descriptions of the planets can be expressed
with the integers, $\Theta = \left\{1, 2, \ldots, 8 \right\}$, with each
logical statement equivalent to a subset of integers
(Figure \ref{fig:discrete_set_logic}).  The procedure is equivalent
for descriptions that can be represented by real numbers, such
as the distance between two objects (Figure \ref{fig:real_set_logic}).

\begin{figure*}
\centering
\subfigure[]{
\begin{tikzpicture}[scale=0.3, thick]
  \draw[color=white] (-25, 0) to (10, 0);

  \node[] at (-15, 0) {The planet has a magnetic field};
  \node[] at (7, 2) {$E_{1}$};

  \fill[color=gray80] (0, 0) circle (25pt) node[color=white] {$\mercury$}; 
  \fill[color=dark] (2, 0) circle (25pt) node[color=white] {$\venus$}; 
  \fill[color=dark] (4, 0) circle (25pt) node[color=white] {$\earth$}; 
  \fill[color=gray80] (6, 0) circle (25pt) node[color=white] {$\mars$}; 
  \fill[color=dark] (8, 0) circle (25pt) node[color=white] {$\jupiter$}; 
  \fill[color=dark] (10, 0) circle (25pt) node[color=white] {$\saturn$}; 
  \fill[color=dark] (12, 0) circle (25pt) node[color=white] {$\uranus$}; 
  \fill[color=dark] (14, 0) circle (25pt) node[color=white] {$\neptune$};  
  
  \node[] at (-15, -5) {The planet has moons};
  \node[] at (7, -3) {$E_{2}$};

  \fill[color=gray80] (0, -5) circle (25pt) node[color=white] {$\mercury$}; 
  \fill[color=gray80] (2, -5) circle (25pt) node[color=white] {$\venus$}; 
  \fill[color=dark] (4, -5) circle (25pt) node[color=white] {$\earth$}; 
  \fill[color=dark] (6, -5) circle (25pt) node[color=white] {$\mars$}; 
  \fill[color=dark] (8, -5) circle (25pt) node[color=white] {$\jupiter$}; 
  \fill[color=dark] (10, -5) circle (25pt) node[color=white] {$\saturn$}; 
  \fill[color=dark] (12, -5) circle (25pt) node[color=white] {$\uranus$}; 
  \fill[color=dark] (14, -5) circle (25pt) node[color=white] {$\neptune$};  
\end{tikzpicture}
}
%
\subfigure[]{
\begin{tikzpicture}[scale=0.3, thick]
  \draw[color=white] (-25, 0) to (10, 0);
  
  \node[align=center] at (-15, 0) {The planet has a magnetic field\\and moons};
  \node[] at (7, 2) {$E_{1} \cap E_{2}$};

  \fill[color=gray80] (0, 0) circle (25pt) node[color=white] {$\mercury$}; 
  \fill[color=gray80] (2, 0) circle (25pt) node[color=white] {$\venus$}; 
  \fill[color=dark] (4, 0) circle (25pt) node[color=white] {$\earth$}; 
  \fill[color=gray80] (6, 0) circle (25pt) node[color=white] {$\mars$}; 
  \fill[color=dark] (8, 0) circle (25pt) node[color=white] {$\jupiter$}; 
  \fill[color=dark] (10, 0) circle (25pt) node[color=white] {$\saturn$}; 
  \fill[color=dark] (12, 0) circle (25pt) node[color=white] {$\uranus$}; 
  \fill[color=dark] (14, 0) circle (25pt) node[color=white] {$\neptune$};  
\end{tikzpicture}
}
%
\subfigure[]{
\begin{tikzpicture}[scale=0.3, thick]
  \draw[color=white] (-25, 0) to (10, 0);
   
  \node[align=center] at (-15, 0) {The planet has a magnetic field\\or moons};
  \node[] at (7, 2) {$E_{1} \cup E_{2}$};

  \fill[color=gray80] (0, 0) circle (25pt) node[color=white] {$\mercury$}; 
  \fill[color=dark] (2, 0) circle (25pt) node[color=white] {$\venus$}; 
  \fill[color=dark] (4, 0) circle (25pt) node[color=white] {$\earth$}; 
  \fill[color=dark] (6, 0) circle (25pt) node[color=white] {$\mars$}; 
  \fill[color=dark] (8, 0) circle (25pt) node[color=white] {$\jupiter$}; 
  \fill[color=dark] (10, 0) circle (25pt) node[color=white] {$\saturn$}; 
  \fill[color=dark] (12, 0) circle (25pt) node[color=white] {$\uranus$}; 
  \fill[color=dark] (14, 0) circle (25pt) node[color=white] {$\neptune$};  
\end{tikzpicture}
}
%
\subfigure[]{
\begin{tikzpicture}[scale=0.3, thick]
  \draw[color=white] (-25, 0) to (10, 0);
   
  \node[align=center] at (-15, 0) {The planet does not\\have a magnetic field};
  \node[] at (7, 2) {$E_{1}^{c}$};

  \fill[color=dark] (0, 0) circle (25pt) node[color=white] {$\mercury$}; 
  \fill[color=gray80] (2, 0) circle (25pt) node[color=white] {$\venus$}; 
  \fill[color=gray80] (4, 0) circle (25pt) node[color=white] {$\earth$}; 
  \fill[color=dark] (6, 0) circle (25pt) node[color=white] {$\mars$}; 
  \fill[color=gray80] (8, 0) circle (25pt) node[color=white] {$\jupiter$}; 
  \fill[color=gray80] (10, 0) circle (25pt) node[color=white] {$\saturn$}; 
  \fill[color=gray80] (12, 0) circle (25pt) node[color=white] {$\uranus$}; 
  \fill[color=gray80] (14, 0) circle (25pt) node[color=white] {$\neptune$};  
\end{tikzpicture}
}
\caption{(a) Descriptions of a discrete system, such as the collection of
planets in our solar system, can be expressed with by the integers, with
any of the planets can be represented by a subset of the integers. 
(b) Conjuction of logical statements is implemented with set intersections, 
(c) disjunction with set unions, and (d) negation with set negation.}
\label{fig:discrete_set_logic}
\end{figure*}

\begin{figure*}
\centering
\subfigure[]{
\begin{tikzpicture}[scale=0.3, thick]
  \draw[color=white] (-27, 0) to (17, 0);

  \node[align=center] at (-15, 0) {The distance is\\less than five inches};
  \node[] at (7.5, 2) {$E_{1}$};

  \draw[|->] (0, 0) -- (14,0) node[right] {$x$};
  \draw[line width=1mm, color=dark] (0, 0) node[] {$\,($} -- (5, 0) node[] {$\!)$};
  
  \node[align=center] at (-15, -5) {The distance is between\\three and seven inches};
  \node[] at (7.5, -3) {$E_{2}$};
  
  \draw[|->] (0, -5) -- (14,-5) node[right] {$x$};
  \draw[line width=1mm, color=dark] (3, -5) node[] {$\,($} -- (7,-5) node[] {$\!)$};

\end{tikzpicture}
}
%
\subfigure[]{
\begin{tikzpicture}[scale=0.3, thick]
  \draw[color=white] (-27, 0) to (17, 0);
  
  \node[align=center] at (-15, 0) {The distance is less than five inches\\and
                                                   between three and seven inches};
  \node[] at (7.5, 2) {$E_{1} \cap E_{2}$};

  \draw[|->] (0, 0) -- (14,0) node[right] {$x$};
  \draw[line width=1mm, color=dark] (3, 0) node[] {$\,($} -- (5, 0) node[] {$\!)$};
\end{tikzpicture}
}
%
\subfigure[]{
\begin{tikzpicture}[scale=0.3, thick]
  \draw[color=white] (-27, 0) to (17, 0);
   
  \node[align=center] at (-15, 0) {The distance is less than five inches\\or
                                                   between three and seven inches};
  \node[] at (7.5, 2) {$E_{1} \cup E_{2}$};

  \draw[|->] (0, 0) -- (14, 0) node[right] {$x$};
  \draw[line width=1mm, color=dark] (0, 0) node[] {$\,($} -- (7, 0) node[] {$\!)$};
\end{tikzpicture}
}
%
\subfigure[]{
\begin{tikzpicture}[scale=0.3, thick]
  \draw[color=white] (-27, 0) to (17, 0);
   
  \node[align=center] at (-15, 0) {The distance is not\\less than five inches};
  \node[] at (7.5, 2) {$E_{1}^{c}$};

  \draw[|->] (0, 0) -- (14, 0) node[right] {$x$};
  \draw[line width=1mm, color=dark] (5, 0) node[] {$\,($} -- (13, 0);
\end{tikzpicture}
}
\caption{Systems whose descriptions can be described by the real numbers 
are expressed exactly as with discrete systems. (a) Descriptions are expressed
as subsets of the real numbers, (b) conjunction with intersections of those
subsets, (c) disjunction with unions of those subsets, and (e) negation with
complements. }
\label{fig:real_set_logic}
\end{figure*}

One extremely important subtlety with this construction is that expressions are 
not unique -- we can express the same logical statements using events in different 
sample spaces.  To see this consider an invertible map from one sample space 
into another, $s : \Theta \rightarrow \Omega$; if every event in $\Theta$ maps 
to an event in  $\Omega$ and vice versa,
%
\begin{align*}
s \! \left( E_{\Theta} \right) &\in \EV{\Omega}
\\
s^{-1} \! \left( E_{\Omega} \right) &\in \EV{\Theta}
\end{align*}
%
then the map is called \emph{measurable}.  The composition of an expression
with a measurable map,
%
\begin{equation*}
s \circ \mathfrak{e} : \mathcal{S} \rightarrow \Theta \rightarrow \Omega,
\end{equation*}
%
is another valid expression.  Hence when a measurable map exists between 
two sample spaces they can be used to to specify the same logical statements,
and we say that they are \emph{equivalent} as they provide equivalent descriptions 
of the given system.  The set of samples spaces equivalent to $\Theta$ is 
denoted $\lfloor \Theta \rfloor$.  \textbf{Mention that equivalent sample
spaces are typically restricted to common topologies, so we can map
discrete to discrete and real to real, but not discrete to real and vice
versa?}

For example, if the sample space is discrete then we can always define an 
equivalent sample space by simply permuting the labels.  It doesn't matter 
if we order the planets by distance from the sun, by diameter, or by any other 
metric: the sample spaces quantify the same information (Figure
\ref{fig:permuting_discrete_spaces}).   Similarly, we can always apply a 
transformation that warps the real numbers to map between real samples spaces 
without compromising our ability to represent statements with events.  One of 
the most common ways this manifests in practice is when our descriptions require 
units -- the information we quantify doesn't depend on whether we express distance 
in {\aa}ngstr\"{o}ms or inches or meters or furlongs, each unit defines a separate 
but equivalent sample space.

\begin{figure*}
\centering
\begin{tikzpicture}[scale=0.3, thick]
  \foreach \i in {1, 2, ..., 8} {
    \node[] at ({2 * \i - 2}, 2) {$\i$};
  }  

  \fill[color=dark] (0, 0) circle (25pt) node[color=white] {$\mercury$}; 
  \fill[color=dark] (2, 0) circle (25pt) node[color=white] {$\venus$}; 
  \fill[color=dark] (4, 0) circle (25pt) node[color=white] {$\earth$}; 
  \fill[color=dark] (6, 0) circle (25pt) node[color=white] {$\mars$}; 
  \fill[color=dark] (8, 0) circle (25pt) node[color=white] {$\jupiter$}; 
  \fill[color=dark] (10, 0) circle (25pt) node[color=white] {$\saturn$}; 
  \fill[color=dark] (12, 0) circle (25pt) node[color=white] {$\uranus$}; 
  \fill[color=dark] (14, 0) circle (25pt) node[color=white] {$\neptune$};  
  
  \draw[->] (0, -1) -- (0, -5);
  \draw[->] (2, -1) -- (4, -5);
  \draw[->] (4, -1) -- (6, -5);
  \draw[->] (6, -1) -- (2, -5);
  \draw[->] (8, -1) -- (14, -5);
  \draw[->] (10, -1) -- (12, -5);
  \draw[->] (12, -1) -- (10, -5);
  \draw[->] (14, -1) -- (8, -5);
  
  \fill[color=dark] (0, -6) circle (25pt) node[color=white] {$\mercury$}; 
  \fill[color=dark] (2, -6) circle (25pt) node[color=white] {$\mars$}; 
  \fill[color=dark] (4, -6) circle (25pt) node[color=white] {$\venus$}; 
  \fill[color=dark] (6, -6) circle (25pt) node[color=white] {$\earth$}; 
  \fill[color=dark] (8, -6) circle (25pt) node[color=white] {$\neptune$};  
  \fill[color=dark] (10, -6) circle (25pt) node[color=white] {$\uranus$};
  \fill[color=dark] (12, -6) circle (25pt) node[color=white] {$\saturn$};  
  \fill[color=dark] (14, -6) circle (25pt) node[color=white] {$\jupiter$}; 

  \foreach \i in {1, 2, ..., 8} {
    \node[] at ({2 * \i - 2}, -8) {$\i$};
  }  
          
\end{tikzpicture}
\caption{Discrete sample spaces are not unique way to express a system 
as we can always permute the arbitrary labels to yield an equivalent sample 
space.  For example, we could order the planets by distance from the sun 
or diameter without affecting our ability to describe the planets themselves.}
\label{fig:permuting_discrete_spaces}
\end{figure*}

Ultimately, there are many ways to express the descriptions of a given 
system and measurable maps translate from one equivalent expression
to another.  Although certain expressions may be more useful than others
in practice, we have to ensure that our final analysis does not depend on 
the irrelevant details of any particular expression.

\subsection{Implications}

Describing simple systems that can expressed with low-dimensional
spaces is a relatively straightforward process, but reasoning about more
complex systems that can be expressed with only high-dimensional
spaces is typically much more subtle.  One elegant way to facilitate the 
description of these complex systems is to decompose each statement
into a composition of many simpler but related \emph{implications}.

An implication is a logical statement conditioned on the another logical
statement.  For example we might not be able to assert that ''a planet
has a magnetic field'' universally, but we can assert that ''if a planet has
a rotating core then it has a magnetic field''.  Logical statements about
both the planets and their cores can always be decomposed into a
statement about one of these systems and an implication from that
system into the other.  Consequently we can build up logical statements
about complex systems by starting with a simple statement and then
adding more and more implications.

If $\Theta$ is a sample space expressing the target system and 
$\Phi$ is a sample space expressing the conditioning systems, then
an implication is map from points in $\Phi$ to events in $\Theta$,
%
\begin{align*}
\iota_{\Theta \mid \Phi} :& \, \Phi \rightarrow \EV{\Theta}
\\
& \, \phi \mapsto \iota_{\Theta \mid \Phi} \! \left( \phi \right).
\end{align*}
%
For any event in the conditioning space, $E_{\Phi} \in \EV{\Phi}$,
the implication defines an event in the joint sample space, $\Theta \times \Phi$, 
as the union of all of the sets implied by each point in $E_{\Phi}$,
%
\begin{equation*}
\bigcup_{\phi \in E_{\Phi} } \iota_{\Theta \mid \Phi} \! \left( \phi \right)
\equiv
E_{\Theta \times \Phi}
\in
\EV{\Theta \times \Phi}.
\end{equation*}

Consequently a logical statement representable in the joint sample
space $\prod_{n = 1}^{N} \Theta_{n}$ can be built up sequentially, starting with 
an event in one component and adding implications one after another,
%
\begin{align*}
& E_{\Theta_{1}} \\
& \iota_{\Theta_{2} \mid \Theta_{1}} \\
& \iota_{\Theta_{3} \mid \Theta_{2}, \Theta_{1}} \\
& \cdots \\
& \iota_{\Theta_{N} \mid \Theta_{N - 1}, \ldots, \Theta_{2}, \Theta_{1}}.
\end{align*}